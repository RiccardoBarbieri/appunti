\chapter{Multi-Paradigm Programming: parser espressioni in Prolog}

Il motore Prolog contiene un suo scanner e un suo parser usato per interpretare la sua sintassi di regole.

L'obiettivo è quello di mappare la nostra sintassi "sopra" alla sua per riutilizzare le strutture già definite.

Cosa abbiamo già:
\begin{itemize}
    \item tre dei quattro operatori sono già \textit{operatori infissi} leciti in Prolog
\end{itemize}

Cosa manca:
\begin{itemize}
    \item l'operatore \texttt{:} è diverso in quanto Prolog usa la barra \texttt{/}
    \item le parentesi tonde sono già intercettate dal parser di Prolog
\end{itemize}

\subsection{Definizione nuovi operatori}
Per definire un nuovo operatore infisso in Prolog basta porre all'inizio della teoria logica la dichiarazione
\begin{minted}[bgcolor=lightgray,framesep=2mm,baselinestretch=1.2,fontsize=\footnotesize]{Prolog}
:- op(livellopriorità, associatività, operatore)
\end{minted}

\begin{itemize}
    \item livellopriorità: numero che esprime la priorità dell'operatore
    \item associatività: atomo che esprime se l'operatore è infisso o pre/postfisso e la sua associatività
    \item operatore: è il nuovo operatore da dichiarare
\end{itemize}
\begin{minted}[bgcolor=lightgray,framesep=2mm,baselinestretch=1.2,fontsize=\footnotesize]{Prolog}
:- op(400, yfx, ':')
\end{minted}

\subsection{Problema delle parentesi tonde}
Non si possono utilizzare parentesi tonde dato che sono utilizzate da Prolog, sostituiamo quindi le tonde con le parentesi quadre, anch'esse già utilizzate da prolog per le liste (già riconosciute e bilanciate).

Il prezzo sarà minimo, in quanto consiste solo nella differenza della forma delle parentesi, esistono inoltre tecniche per mascherarle.

\subsection{Espressioni in Prolog}
La semantica denotazionale definita esprime già tutte le regole "pattern oriented", sono direttamente trasferirle in Prolog.


\begin{minted}[bgcolor=lightgray,framesep=2mm,baselinestretch=1.2,fontsize=\footnotesize]{Prolog}
fExpr(Term) :- fTerm(Term).
fExpr(Exp+Term) :- fExpr(Exp), fTerm(Term).
fExpr(Exp-Term) :- fExpr(Exp), fTerm(Term).
fTerm(Factor) :- fFactor(Factor).
fTerm(Term*Factor) :- fTerm(Term), fFactor(Factor).
fTerm(Term:Factor) :- fTerm(Term), fFactor(Factor).
fFactor([Exp]) :- fExpr(Exp).
fFactor(num) :- number(Num).
\end{minted}

Le funzioni sono solo sintassi, per ora non hanno nessun significato.

\subsection{Dal riconoscitore al valutatore}
Per sintetizzare il valutatore è necessario:
\begin{itemize}
    \item estendere la testa delle regole aggiungendo un \textit{argomento per restituire il risultato}
    \item estendere il corpo delle regole, ricavando i \textit{valori} e \textit{combinandoli} in modo appropriato alla semantica
\end{itemize}

Per il trattamento dei numeri, Prolog utilizza soltanto il concetto di numeri reali, e attraverso il predicato \texttt{is} si può applicare la semantica dei numeri naturali.

\begin{minted}[bgcolor=lightgray,framesep=2mm,baselinestretch=1.2,fontsize=\footnotesize]{prolog}
evalExpr(Term, Value) :- evalTerm(Term, Value).
evalExpr(Exp+Term, Value) :- evalExpr(Exp, Value1),
                             evalTerm(Term, Value2),
                             Value is Value1 + Value2.
evalExpr(Exp-Term, Value) :- evalExpr(Exp, Value1),
                             evalTerm(Term, Value2),
                             Value is Value1 - Value2.
evalTerm(Factor, Value) :- evalFactor(Factor, Value).
evalTerm(Term*Factor, Value) :- evalTerm(Term, Value1),
                                evalFactor(Factor, Value2),
                                Value is Value1 * Value2.
evalTerm(Term:Factor, Value) :- evalTerm(Term, Value1),
                                evalFactor(Factor, Value2),
                                Value is Value1 / Value2.
evalFactor([Exp], Value) :- evalExpr(Exp, Value).
evalFactor(Num, Num) :- number(Num).
\end{minted}

\section{Parser ibrido con 2p-kt}
Java si occupa di:
\begin{itemize}
    \item visualizzare finestre di dialogo per richiedere stringa di input
    \item sostituire parentesi tonde con quadre
    \item \textit{creare} un \textbf{motore Prolog}, configurandolo con l'opportuna \textbf{teoria} e \textbf{interrogarlo} con l'espressione
    \item recuperare e visualizzare il risultato della valutazione
\end{itemize}
tuProlog si occupa di
\begin{itemize}
    \item interpretare e valutare l'espressione data
\end{itemize}

\subsection{Primi passi per utilizzare il Prolog engine}

Leggere la teoria logica (input o file)
\begin{minted}[bgcolor=lightgray,framesep=2mm,baselinestretch=1.2,fontsize=\footnotesize]{java}
ClausesReader theoryReader = ClausesReader.getWithDefaultOperators();
Theory theory = theoryReader.readTheory(input);
\end{minted}

Ottenere una istanza di \texttt{SolverFactory}, per ottenere il un solver adeguato
\begin{minted}[bgcolor=lightgray,framesep=2mm,baselinestretch=1.2,fontsize=\footnotesize]{java}
SolverFactory solverFactory = ClassicSolverFactory.INSTANCE;
\end{minted}

Costruire tramite la factory il solver, passando la teoria
\begin{minted}[bgcolor=lightgray,framesep=2mm,baselinestretch=1.2,fontsize=\footnotesize]{java}
Solver solver = solverFactory.solverWithDefaultBuiltins(theory);
\end{minted}

\subsection{TermParser}
Il \texttt{TermParser} è il componente che effettua il parsing dei termini, deve quindi riconoscere tutti gli operatori, anche quelli definiti custom.

É possibile costruire un \texttt{TermParser} con i giusti operatori "a mano" oppure è possibile ottenere dal solver le definizioni trovate.

\subsubsection{Metodo 1}
\begin{minted}[bgcolor=lightgray,framesep=2mm,baselinestretch=1.2,fontsize=\footnotesize]{java}
OperatorSet myOpSet = OperatorSet.STANDARD.plus(new Operator(":", Specifier.YFX, 400));
TermParser termParser = TermParser.withOperators(myOpSet);
\end{minted}

\subsubsection{Metodo 2}

\begin{minted}[bgcolor=lightgray,framesep=2mm,baselinestretch=1.2,fontsize=\footnotesize]{java}
TermParer termParser = TermParser.withOperators(solver.getOperators())
\end{minted}

Per creare la query a partire dalla rappresentazione testuale
\begin{minted}[bgcolor=lightgray,framesep=2mm,baselinestretch=1.2,fontsize=\footnotesize]{java}
Struct query = termParser.parseStruct("evalExpr(" + queryText + ", Result)");
\end{minted}
















































