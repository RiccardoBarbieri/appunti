\chapter{Estensione: assegnamenti, ambienti, sequenze}

\section{Assegnamento}
Ogni linguaggio di programmazione introduce le nozioni di \textbf{variabili} e \textbf{assegnamento}.

L'assegnamento non è una uguaglianza o equazione, ma denota l'azione di prendere la destra dell'uguale e metterla all'interno della sinistra dell'uguale, considerando come valori le variabili sulla destra e contenitori le variabili sulla sinistra.

Si deduce che il simbolo di variabile ha significato diverso tra la destra e la sinistra dell'operatore.

\subsection{L-Value vs R-Value}
Si definisce quindi la distinzione tra l-value e r-value:
\begin{itemize}
    \item \textbf{l-value}: il significato della variabile a sinistra è la variabile \textit{in quanto tale}
    \item \textbf{r-value}: il significato della variabile a destra è il \textit{contenuto} della variabile
\end{itemize}

Sorge inoltre la questione dell'\textit{assegnamento distruttivo/non distruttivo}, è possibile cambiare il valore associato in precedenza a un simbolo di variabile?

Nei linguaggi \textit{imperativi} l'assegnameto è distruttivo, nei linguaggi \textit{logici} si ha una trasparenza referenziale dove un simbolo ha sempre lo stesso valore.

\section{Environment}
Per esprimere al meglio la semantica dell'assegnamento, occorre introdurre il concetto di \textbf{environment}, inteso come \textit{insieme di coppie} \texttt{(simbolo, valore)}

































