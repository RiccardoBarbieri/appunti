\chapter{Multi-Paradigm Programming con JavaScript}

\section{Funzioni}

Le \textbf{funzioni} sono introdotte dalla keyword \texttt{function}:
\begin{itemize}
    \item il nome è \textit{opzionale}, possibili funzioni anonime
    \item sintassi alternativa: \textit{lambda expression}
    \item \textit{function expression} vs \textit{function declaration}
    \item non si usa la keyword \texttt{void}
\end{itemize}

Parametri privi di dichiarazione di tipo
\begin{minted}[bgcolor=lightgray,framesep=2mm,baselinestretch=1.2,fontsize=\footnotesize,escapeinside=||,mathescape=true]{js}
// funzione
function sum(a, b) { return a + b }
// procedura no return
function printSum(a, b) {
    document.write(a + b)
}
// funzione anonima
var sum = (a, b) => a + b
\end{minted}

A differenza di molti altri linguaggi, i parametri \textit{attuali} possono non corrispondere in numero ai parametri \textit{formali}:
\begin{itemize}
    \item se sono più, extra ignorati
    \item se sono meno, mancanti \texttt{undefined}
\end{itemize}

\subsection{Funzioni come first class entities}
In javascript, una funzione è una entità-oggetto, manipolabile come ogni altro tipo:
\begin{itemize}
    \item assegnabile a variabili
    \begin{itemize}
        \item[] \texttt{var f = function(x) \{ return x / 10 \}}
        \item[] \texttt{var f = x => x / 10}
    \end{itemize}
    \item definita e usata al "volo" con un \textit{function literal}
    \begin{itemize}
        \item[] \texttt{var z = function(y) \{ return y + 1 \}(8)}
        \item[] \texttt{var z = (y = > y + 1)(8)}
    \end{itemize}
    \item passata come argomento
    \begin{itemize}
        \item[] \texttt{var p = ff(f)}
    \end{itemize}
    \item restituita da funzione factory
    \begin{itemize}
        \item[] \texttt{var f = fgen(...)}
    \end{itemize}
\end{itemize}

\subsection{Function expression vs function declaration}
La \textbf{function expression} assegna una funzione a una variabile
\begin{itemize}
    \item il nome della funzione non è essenziale
    \item lo scope del nome, se presente, è il \textit{corpo della funzione} (utile in chiamate ricorsive)
\end{itemize}
\begin{minted}[bgcolor=lightgray,framesep=2mm,baselinestretch=1.2,fontsize=\footnotesize,escapeinside=||,mathescape=true]{js}
var f = function g(x) { return x / 10 }
g(32) //errore, nome g non definito
\end{minted}

La \textbf{function declaration} introduce una funzione senza assegnarla a una variabile
\begin{itemize}
    \item il nome è essenziale
    \item lo scope è l'\textit{ambiente di definizione} della funzione
\end{itemize}
\begin{minted}[bgcolor=lightgray,framesep=2mm,baselinestretch=1.2,fontsize=\footnotesize,escapeinside=||,mathescape=true]{js}
function g(x) { return  x / 10 }
g(32) //nome g definito
\end{minted}

\subsection{Funzioni innestate e chiusure}
Si possono definire funzioni dentro altre, nasce quindi la discussione delle \textbf{chiusure}, javascript segue la chiusura \textit{lessicale}.

Una chiusura nasce quando una funzione innestata fa riferimento a \textit{variabili della funzione esterna}.

\subsection{Currying}
Un caso interessante di chiusura è la possibilità di simulare una funzione a N argomenti con N funzioni a 1 argomento.

Si può esprimere una funzione come questa
\begin{minted}[bgcolor=lightgray,framesep=2mm,baselinestretch=1.2,fontsize=\footnotesize,escapeinside=||,mathescape=true]{js}
function sum(a, b) { return a + b }
var result = sum(3, 4)
\end{minted}
in una forma diversa, basata su una funzione "esterna" con argomento \texttt{a}, che restituisce una funzione "interna" con argomento \texttt{b}.
\begin{minted}[bgcolor=lightgray,framesep=2mm,baselinestretch=1.2,fontsize=\footnotesize,escapeinside=||,mathescape=true]{js}
function sum(a) { return function(b) { return a + b } }
var result = sum(3)(4)
\end{minted}

Questa possibilità si chiama \textbf{currying}.

É possibile utilizzare anche la definizione tramite lambda con la seguente sintassi
\begin{minted}[bgcolor=lightgray,framesep=2mm,baselinestretch=1.2,fontsize=\footnotesize,escapeinside=||,mathescape=true]{js}
sum = a => b => a + b
var result = sum(3)(4)
\end{minted}
Il currying è concettualmente interessante perché indica che l'unico "ingrediente" fondamentale per esprimere qualunque funzione sono le \textit{funzioni a un argomento}.

\subsection{Utilizzi chiusure}

\subsubsection{Rappresentare uno stato privato e nascosto}
Si può ottenere una \textit{proprietà privata} tramite una chiusura, mappando lo stato su un argomento della funzione "generatrice"
\begin{minted}[bgcolor=lightgray,framesep=2mm,baselinestretch=1.2,fontsize=\footnotesize,escapeinside=||,mathescape=true]{js}
function incBy2From(x) {
    return function() { return x += 2 }
}
\end{minted}

Da ogni invocazione di \texttt{incBy2From}, nasce un nuova funzione, che ha uno "stato" interno privato e utilizza come punto di partenza il parametro passato al generatore.

Altro esempio, generatore di contatori
\begin{minted}[bgcolor=lightgray,framesep=2mm,baselinestretch=1.2,fontsize=\footnotesize,escapeinside=||,mathescape=true]{js}
function genContatore(){
    var contati=0;
    function tick() { return contati++; }
    function num() { return contati;}
    //metodo per restituire più funzioni di accesso
    return { num, tick };
}
\end{minted}

\subsubsection{Realizzare un canale di comunicazione privato}
Si può ottenere un canale di comunicazione privato, mettendo in una chiusura sia lo \textbf{stato} che i due \textbf{metodi accessor}.

Si utilizza un array di due funzioni per restituirli
\begin{minted}[bgcolor=lightgray,framesep=2mm,baselinestretch=1.2,fontsize=\footnotesize,escapeinside=||,mathescape=true]{js}
function myChannel() {
    var msg = "bla bla"
    return [ function(x) { msg = x; },    //set
            function() { return msg; } ]  //get
}

var channel = myChannel()
var msg1 = channel[1]() //recupera "bla bla"
channel[0]("bruuh")      //setta il nuovo msg
var msg2 = channel[1]() //recupera "bruuh"
\end{minted}

Versione con restituzione di object literal con due accessor
\begin{minted}[bgcolor=lightgray,framesep=2mm,baselinestretch=1.2,fontsize=\footnotesize,escapeinside=||,mathescape=true]{js}
function canale() {
    var msg = "";
    function set(m) { msg = m }
    function get() { return msg }
    return { set, get }
}

var ch = canale()

document.writeln(ch.set("hello")); // undefined
document.writeln(ch.set("world")); // undefined
document.writeln(ch.get());        // world
\end{minted}

\subsubsection{Realizzer nuove strutture di controllo}
Una funzione di secondo

























