\chapter{Basi di programmazione funzionale}

Idee chiave del paradigma funzionale:
\begin{itemize}
    \item distinzioni variabili / valori (\textbf{var} e \textbf{val})
    \item costrutti come espressioni: \textit{everything is an expression}
    \item collezione e oggetti immodificabili: \textit{compute by synthesis}
    \item funzioni sono \textbf{first-class entitites}
    \begin{itemize}
        \item chiusure, lambda expressions
        \item \textbf{lazy} evaluation vs \textbf{eager} evaluation
    \end{itemize}
    \item abolizione tipi primitivi: \textit{everything is an object}
    \item abolizione parti statiche in favore di singleton \textit{companion}
\end{itemize}

\section{Uniformità}

\subsection{Everything is an object}
I linguaggi a oggetti tradizionali, pur chiamandosi a oggetti, si basano su tipo primitivi che non sono oggetti.

Questa caratteristica causa disuniformità che necessita di trattamento:
\begin{itemize}
    \item tipi primitivi passati per \textbf{valore} vs oggetti passati per \textbf{riferimento}
    \item collezioni di tipi primitivi: non ammesse o classi wrapper
\end{itemize}

I linguaggi blended ripuliscono lo spazio concettuale eliminando i tipi primitivi, si distingue però tra \textbf{classi valore} (entità immodificabili) e \textbf{classi riferimento} (classi classiche).

\subsection{Don't be static}
I linguaggi a oggetti tradizionali, pur chiamandosi oggetti, si basano sulle \textit{classi}, che non sono oggetti.

I linguaggi blended cancellano le \textit{parti statiche}: il loro ruolo è svolto da un singleton, il \textbf{comapanion object}:
\begin{itemize}
    \item in Scala, il co ha lo stesso nome della classe
    \item in Kotlin, può anche avere nome diverso (ma nello stesso file)
\end{itemize}

\section{Oggetti immodificabili}

\subsection{Distinzioni \texttt{var} vs \texttt{val}}
Nel paradigma imperativo, basato sulla modifica dei valori, si generano programmi difficili da leggere e manutenere dato che i simboli cambiano continuamente significato.

Il paradigma funzionale si basa sul concetto opposto: i simboli \textit{mantengono il loro significato}.

I moderni linguaggi blended uniscono i due mondi:
\begin{itemize}
    \item \texttt{var} denota variabili
    \item \texttt{val} denota valori
\end{itemize}

Proprietà introdotte da \texttt{val} hanno solo getter e in certi contesti il suo utilizzo è semplificato o obbligatorio.

\subsection{Null safety}
L'introduzione del null mina alla base la correttezza dei programmi.

Moderni linguaggi introducono forme di \textit{Null safety}:
\begin{itemize}
    \item vietando che un riferimento sia \texttt{null}
    \item può essere etichettato esplicitamente come tale
\end{itemize}

\subsection{Collezioni immodificabili}
Completa

\subsubsection{Expressions everywhere}
I linguaggi imperativi distinguono tra \textit{istruzioni} e \textit{espressioni}
\begin{itemize}
    \item \texttt{for, while, if, throw} sono istruzioni: non denotano un valore ma denotano azioni
    \item 
\end{itemize}

La presenza di istruzioni induce a esprimere computazioni mettendole in sequenza che implica variabili di appoggio.

Nei linguaggi blended, tutte le istruzioni sono sostituite da espressioni, tutto produce valori.

\section{Funzioni come first-class entities}
Ogni linguaggi di programmazione offre la possibilità di creare funzioni, nei linguaggi blended le funzioni sono anche \textit{first-class entities}.

Nei linguaggi blended, le funzioni hanno queste caratteristiche:
\begin{itemize}
    \item possono essere \textit{assegnate a variabili} (tipo funzione)
    \item possono essere \textit{passate come argomento} a un'altra funzione
    \item possono essere \textit{restituite da un'altra funzione}
    \item possono essere \textit{definite e usate al volo} come valori di ogni altro tipo
\end{itemize}

Con questi concetti, è possibile definire \textit{funzioni di ordine superiore}, che sono funzioni che manipolano/generano altre funzioni, e si può estendere il concetto a funzione di secondo, terzo ... ordine.

Assegnare e restiture funzioni implica la necessità di avere \textbf{tipi funzione} che \textit{incapsulano il comportamento} delle funzioni.

Esempi in javascript:
\begin{minted}[bgcolor=lightgray,framesep=2mm,baselinestretch=1.2,fontsize=\footnotesize,escapeinside=||,mathescape=true]{javascript}
// keyword function definisce oggetti-funzioni (anonimi)
var f = function (z) { return z * z; }
// funzione può riceverne un'altra come argomento
var ff = function (f, x) { return f(x); }
// si può passare una funzione come argomento a un'altra
var res = loop( function() { i++ }, 10 );
// si può restituire un risultato funzione
function fr() { return function(r) { return r + 10 } }
// si può creare una funzione da stringhe
square = new Function("x", "return x * x")
\end{minted}

In javascript esiste anche una sintassi per definire funzioni senza la keyword \textit{function}
\begin{minted}[bgcolor=lightgray,framesep=2mm,baselinestretch=1.2,fontsize=\footnotesize,escapeinside=||,mathescape=true]{javascript}
var f = z => z * z

var ff = (f, x) => f(x)

var res = loop( () => i++, 10)

fr = () => r => r + 10
\end{minted}

\section{Variabili libere e chiusure}
Se un linguaggi ammette la definizione di \textit{funzioni con variabili libere} (non definite localmente), tali funzioni dipendono dal contesto circostante.

Sono necessari criteri per dare significato alle variabili libere:
\begin{itemize}
    \item \textbf{chiusura lessicale}
    \item \textbf{chiusura dinamica}
\end{itemize}

\subsection{Funzioni e chiusure}
Se un linguaggi supporta funzioni come first-class entities, la presenza di variabili libere determina la nascita della nozione di \textbf{chiusura}:
\begin{itemize}
    \item un \textit{oggetto funzione} ottenuto \underline{chiudendo} rispetto a un certo contesto una definizione di funzione che aveva variabili libere
    \item il prodotto finale dell'atto di chiudere le variabili libere rispetto a un certo contesto d'uso
\end{itemize}

\subsubsection{Tempo di vita delle variabili}
In presenza di chiusure, il tempo di vita delle variabili di una funzione \textit{non coincide più necessariamente con quello della funzione che le contiene}.

Il modello computazionale deve tenere conto delle variabili libere, non più allocate semplicemente sullo stack.

Una variabile locale a una funzione \textit{esterna} può essere indispensabile al funzionamento della funzione più \textit{interna}, il tempo di vita delle variabili locali che fanno parte di una chiusura è pari a quello della chiusura, anche se la funzione che la definisce \textit{termina prima}.

Queste variabili vengono allocate sull'\textit{heap}.

\subsubsection{Esempio}
\begin{minted}[bgcolor=lightgray,framesep=2mm,baselinestretch=1.2,fontsize=\footnotesize,escapeinside=||,mathescape=true]{javascript}
function ff(f, x) {
    return function(r) { return f(x) + r; }
}
\end{minted}

Per poter operare, tale nuova funzione necessita di \texttt{f}, \texttt{x} (e \texttt{r})
\begin{minted}[bgcolor=lightgray,framesep=2mm,baselinestretch=1.2,fontsize=\footnotesize,escapeinside=||,mathescape=true]{javascript}
var f1 = ff( Math.sin, .8 )
var f2 = ff( function(q) { return q * q }, .8 )
\end{minted}

\subsubsection{Utilità delle chiusure}
\begin{itemize}
    \item rappresentare e gestire uno stato privato e nascosto
    \begin{itemize}
        \item le variabili della funzione esterna, mantenute vive dalla chiusura, restano comunque invisibili fuori da essa
    \end{itemize}
    \item creare una comunicazione nascosta
    \begin{itemize}
        \item definendo più funzioni nella stessa chiusura, esse condividono uno stato nascosto, usabile per comunicare privatamente
    \end{itemize}
    \item definire nuove strutture di controllo
    \begin{itemize}
        \item la funzione esterna esprime il controllo, mentre quella ricevuta come argomento esprime il corpo da seguire
    \end{itemize}
    \item riprogettare/semplificare API e framework di largo uso
    \begin{itemize}
        \item metodi parametrici che ricevono comportamento come argomento
    \end{itemize}
\end{itemize}

\subsection{Criteri di chiusura}

Occorre stabilire un criterio su \textit{come e dove cercare} una definizione per le variabili libere da chiudere.

In presenza di funzioni innestate, si apre una questione:
\begin{itemize}
    \item il testo del programma contiene una catena di \textit{ambienti di definizione innestati}, \textbf{catena lessicale}
    \item l'attivazione delle funzioni crea a runtime una catena di \textit{ambienti attivi}, \textbf{catena dinamica} che riflette l'ordine delle chiamate
\end{itemize}

\subsubsection{Esempio}
\begin{minted}[bgcolor=lightgray,framesep=2mm,baselinestretch=1.2,fontsize=\footnotesize,escapeinside=||,mathescape=true]{javascript}
var x = 20;

function provaEnv(z) { return z + x }

function testEnv() {
    var x = -1;
    return provaEnv(18);
}
\end{minted}
Quale \texttt{x} viene considerata? 

Se si considera la catena lessicale si utilizza la \texttt{x = 20}, la definizione più vicina in senso \textit{lessicale}.

Se si considera la catena dinamica si utilizza la \texttt{x = -1}, la definizione più vicina nel senso \textit{dimanico}.

Stabilire il significato di \texttt{x} in \texttt{provaEnv} seguendo la \textit{sequenza delle chiamate} o la \textit{struttura del programma} non è la stessa cosa.

Se si segue la catena lessicale, il modello computazionale adotta la \textbf{chiusura lessicale}:
\begin{itemize}
    \item CONTRO: vincola a priori un simbolo a un certo legame
    \item PRO: permette di leggere e capire un programma senza ricostruire la sequenza delle chiamate
\end{itemize}

Se si segue la catena dinamica, il modello computazionale adotta la \textbf{chiusura dinamica}:
\begin{itemize}
    \item PRO: massima dinamicità, simboli legati alla specifica situazione
    \item CONTRO: difficoltà a tracciare il reale valore di una variabile
\end{itemize}

\section{Modelli computazionali per la valutazione di funzioni}
Ogni linguaggio che introduca funzioni deve prevedere un modello computazionale per la loro \textbf{valutazione}.

Tale modello deve specificare:
\begin{itemize}
    \item quando si valutano i \textbf{parametri}
    \item cosa si \textbf{passa} alla funzione
    \item come si \textbf{attiva} la funzione
\end{itemize}

Il modello più usato è il \textbf{modello applicativo}
\begin{itemize}
    \item si valutano \textit{subito} i parametri
    \item si passa il \textit{valore} (tranne certi tipi)
    \item tipicamente \textit{chiamandola} e \textit{cedendo il controllo} (modello sincrono)
\end{itemize}

Questo modello è molto usato perché efficiente e semplice da comprendere, tuttavia esistono alcune inefficienze che si possono creare:
\begin{itemize}
    \item valutare sempre i parametri può essere ridondante se un parametro non serve realmente nella funzione
    \item se la valutazione dei parametri da errore questo causa un fallimento di tutta la funzione
\end{itemize}

\subsection{Modello applicativo - fallimenti evitabili}
\begin{minted}[bgcolor=lightgray,framesep=2mm,baselinestretch=1.2,fontsize=\footnotesize,escapeinside=||,mathescape=true]{java}
class Esempio {
    static void printOne(boolean cond, double a, double b){
        if (cond) System.out.println("result = " + a);
        else System.out.println("result = " + b);
    }
    
    public static void main(String[] args) {
        int x=5, y=4;
        printOne(x>y, 3+1, 3/0); //ArithmeticException: div by zero
    }
}
\end{minted}
In questo esempio la valutazione dei parametri attuali causa eccezione, ma la chiamata non sarebbe fallita se i parametri non fossero valutati a priori.

In linguaggi che adottano un modello diverso questo problema non si sarebbe verificato.

\subsection{Modello Call-By-Name}
Il \textbf{modello normale}, adotta un \textit{meccanismo di valutazione} noto come \textbf{Call-By-Name} in cui:
\begin{itemize}
    \item i parametri vengono valutati solo \textbf{al momento dell'uso}, quindi solo se servono
    \item a tal fine vengono passati non dei valori ma degli \textbf{oggetti computazionali}
    \item un parametro ricevuto \textit{può non essere mai calcolato}
\end{itemize}

\subsubsection{Svantaggi di call-by-name}
Nonostante i vantaggi concettuali, è meno efficiente in casi normali e più comuni rispetto al modello applicativo:
\begin{itemize}
    \item i parametri possono esser calcolati più volte
    \item richiede più risorse a runtime
    \item richiede una vm capace di \textit{lazy runtime}
\end{itemize}

\subsubsection{Simulare il modello call-by-name}
Il modello call by name può essere facilmente \textit{simulato} nei linguaggi con funzioni come first class entities.

Questo si può ottenere passando alla funzione \textit{oggetti computazionali} che, quando eseguiti, producono i valori desiderati: si passano alle funzioni delle funzioni che ritornano il valore quando invocate, sostituendo tutti gli usi di un parametro con chiamate a funzione.

Esempo javascript in stile call-by-name
\begin{minted}[bgcolor=lightgray,framesep=2mm,baselinestretch=1.2,fontsize=\footnotesize,escapeinside=||,mathescape=true]{js}
var f = function(flaf, x) {
    return (flag() < 0) ? 5 : x();
}
var b = f(function(){return 3}, function(){abort()});
document.write("result = " + b);
\end{minted}

\subsubsection{Ricostruzione in Java e C\#}
In Java <=7 non è possibile, in quanto le funzioni non sono entità di prima classe.

Da Java 8, le lambda expression catturano funzioni, ma le funzioni non costituiscono un tipo autonomo.

In  C\#, le lambda expression catturano funzioni e i delegati esprimono i veri \textit{tipi funzione}.

In Java 7 la call-by-name può essere implementata definendo una interfaccia che definisce un metodo \texttt{invoke} per ottenere il valore.

In Java 8, con le lambda, si può emulare in modo migliore questo comportamento, ma comunque non in modo pulito come in linguaggi funzionali:
\begin{minted}[bgcolor=lightgray,framesep=2mm,baselinestretch=1.2,fontsize=\footnotesize,escapeinside=||,mathescape=true]{java}
static void printTwo(boolean cond, IntSupplier a, IntSupplier b) {
    if (cond) System.out.println("result = " + a.getAsInt());
    else System.out.println("result = " + b.getAsInt());
}

public static void main(String[] args) {
    int x = 5, y = 4;
    printTwo(x > y, () -> 3+1, () -> 3/0);
}
\end{minted}
Il miglioramento avviene soltanto lato chiamante, si hanno nomi di tipi particolari, ovvero le interfacce funzionali, come \texttt{IntSupplier}.


In C\# i delegati completano l'opera
\begin{minted}[bgcolor=lightgray,framesep=2mm,baselinestretch=1.2,fontsize=\footnotesize,escapeinside=||,mathescape=true]{csharp}
class Esempio {
    delegate double MyFunction(); //vero tipo funzione
    static void printTwo(bool cond, MyFunction a, MyFunction b) {
        if (cond) System.Console.WriteLine("result = " + a());
        else System.Console.WriteLine("result = " + b());
    }
    public static void Main(string[] args) {
        int x = 5, y = 4, a = 0; //var a = 0 per valutazione ottimizzata C\#
        printTwo(x > y, () => 3+1, () => 3/a);
\end{minted}

\subsubsection{Ricostruzione in Kotlin e Scala}
In Kotlin e Scala si può ricostruire la call-by-name come in C\#, ma in Scala non è necessario, in quanto il linguaggio supporta nativamente le call by name.

\begin{minted}[bgcolor=lightgray,framesep=2mm,baselinestretch=1.2,fontsize=\footnotesize,escapeinside=||,mathescape=true]{scala}
object CallByName {
    def printTwo(cond: Boolean, a: => Int, b: => Int): Unit = {
        if (cond) println("result = " + a);
        else println("result = " + b);
    }
    
    def main(args: Array[String]) {
        val x=5;
        val y=4;
        printTwo(x>y, 3+1, 3/0);
    }
}
\end{minted}
