\chapter{Strumenti per la generazione automatica di riconoscitori LL}

Esistono moltissimi strumenti chiamati \textit{parser generator}, o compiler compiler, questi, data una grammatica (annotata), producono automaticamente il riconsocitore in un linguaggi prescelto.

Alcuni seguono l'approccio LL, altri quello LR:
\begin{itemize}
    \item LL è efficiente e facilita l'aggiunta di azioni semantiche
    \item LR è più potente (vedi ricorsione sinistra) ma più complesso
\end{itemize}

Oltre alla generazione del perser tuttavia, è necessario integrarsi con strumenti di sviluppo, come impostare validazione, segnalazioni di errori, content assists, quick fix e altro.

\section{Domain-Specific Languages}
Al contrario dei GPL, General Purpose Languages, i \textbf{DSL} sono realizzati per essere utilizzati in \textbf{specifici domini applicativi}, non mirano a \textit{risolvere ogni problema} ma a offrire soluzioni efficaci per uno specifico ambito.

Esistono strumenti a supporto dei DSL, utili per la \textit{generazione automatica} dei DSL e del supporto, il più diffuso è \texttt{Xtext}.





















