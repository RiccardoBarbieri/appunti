\chapter{Lambda calcolo}

\subsubsection{Definizione}

\begin{mdframed}[topline=false,bottomline=false,rightline=false]
\begin{align*}
&L ::= \lambda x.L | x | LL\\
&(\lambda x.L_b)L_a \rightarrow L_b[L_a/x]
\end{align*}
$L$ può esprime una qualsiasi \textbf{struttura dati} e qualsiasi \textbf{algoritmo}.

\noindent
$\rightarrow$ rappresenta una \textbf{trasformazione atomica}, basata sull'idea della sostituzione di testo che risulta nel termine $L_b$ con, al suo interno, $x$ sostituite da $L_a$.
\end{mdframed}

\subsubsection{Sintassi}
\begin{equation*}
L,M,N ::= \lambda x.L | x | LL
\end{equation*}

Un lambda termine può essere:
\begin{itemize}
    \item $x$, una \textit{variabile} ($\lambda$ e $.$ sono simboli terminali)
    \item $\lambda x.L$, una \textit{funzione}, dove $L$ è il corpo (altro lambda-termine)
    \item $LM$, \textit{applicazione di una funzione}
\end{itemize}

\subsubsection{Ambiguità grammaticale}
La grammatica di una lambda è \textit{ambigua}, come si interpreta $\lambda x.xy$?
\begin{itemize}
    \item come $(\lambda x.x)y$ (applicare $y$ a funzione $\lambda x.x$)
    \item o come $\lambda x.(xy)$
\end{itemize}

Per disambiguare si utilizzano parentesi o associatività a sinistra, $\lambda x.\lambda y.xyx$ si legge come $\lambda x.(\lambda y.(xy)x)$

\subsubsection{Semantica}
La semantica è definita tramite \textit{regole di trasformazione}:
\begin{equation*}
(\lambda x.L)M \rightarrow  L[M/x]
\end{equation*}

Si identifica un pattern del tipo $(\lambda x.L)M$ e lo si trasforma nel lambda-termine $L[M/x]$, sostituendo, nella funzione $L$, ogni occorrenza $x$ con $M$.

\subsubsection{Esempio}
$(\lambda x.xx)(\lambda y.(\lambda z.yz))$ si riduce a $(\lambda y.(\lambda z.yz))(\lambda y.(\lambda z.yz))$, in quanto la lambda $(\lambda x.xx)$ duplica il suo parametro formale $x$, che in questo caso assume il valore di $(\lambda y.(\lambda z.yz))$.

\subsubsection{Forme normalizzate}
Un lambda termine è in \textbf{forma normale} se non si possono applicare altre riduzioni.

Si dice che un lambda termine è:
\begin{itemize}
    \item \textbit{fortemente} normalizzabile, se qualunque sequenza di riduzioni porta a una forma normale
    \item \textit{debolmente} normalizzabile, se esiste \textit{almeno una} sequenza di riduzioni che porta a una forma normale
    \item \textit{non} normalizzabile, se non si arriva mai a una forma normale
\end{itemize}

\section{Teorema di Church-Rosser - Terminazione delle computazioni}
\begin{mdframed}[topline=false,bottomline=false,rightline=false]
Ogni lambda ha al più \textbf{una} forma normale, questa può essere interpretata come risultato.

Il lambda calcolo è deterministico.
\end{mdframed}

\section{Strategie di riduzione}
L'idea di ridurre il più possibile, introduce il concetto di \textit{strategia di riduzione}, scegliendo la strategia errata, le sostituzioni potrebbero non terminale.

\subsection{Principi}
Adottare una o l'altra strategia, si possono ottenere sostituzioni \textit{infinite} o \textit{finite}.

Due principi alternativi:
\begin{itemize}
    \item \textbf{eager evaluation}: privilegia la valutazione dell'argomento, valuta l'argomento il prima possibile
    \item \textbf{lazy evaluation}: privilegia l'applicazione dell'argomento alla funzione, valuta l'argomento il più tardi possibile
\end{itemize}

Da questi principi sono state sviluppate varie strategia, usate nei vari linguaggi.

\subsection{Panoramica}
\begin{itemize}
    \item strategie \textit{eager} (strict):
    \begin{itemize}
        \item applicative order (rightmost innermost)
        \begin{itemize}
            \item prima riduce poi applica gli argomenti
            \item può non trovare la forma normale
        \end{itemize}
        \item call by value
        \item call by reference
        \item call by copy-restore
    \end{itemize}
    \item strategie \textit{lazy} (non-strict)
    \begin{itemize}
        \item normal order (leftmost outermost)
        \begin{itemize}
            \item prima riduce la lambda più a sinistra
            \item normale perché se esiste una forma normale, la trova
        \end{itemize}
        \item call by name
        \item call by need
        \item call by macro
    \end{itemize}
\end{itemize}


\subsection{Strategie eager}

\subsubsection{Applicative order}
Prima valuta gli argomenti, poi li applica alle funzioni. Tenta di ridurre il più possibile i termini dentro la funzione prima di applicare gli argomenti.
\textbf{Non normalizzante}.

\subsubsection{Call by value}
Non riduce i termini prima di applicare gli argomenti. É una famiglia di strategia che differiscono per l'ordine di valutazione degli argomenti (left to right, right to left, undefined).
Se il valore è un puntatore, si ha lo stesso effetto di call-by-reference.

\subsubsection{Call by reference}
La funzione riceve un \textit{riferimento implicito} all'argomento del chiamante, anziché una copia dello stesso.

\subsubsection{Call by copy-restore}
Caso speciale della call-by-reference adatto al multi-threading: ogni funzione riceve una copia privata dell'argomento e lo ricopia indietro alla fine, evita interferenze ma solo l'ultimo valore sopravvive.

\subsection{Strategie lazy}

\subsubsection{Normal order}
Applica gli argomenti alle funzioni prima di ridurli, riducendo per prima la lambda più a sinistra e considerando come corpo ciò che compare a destra.
































