\chapter{Il nucleo di un sistema multiprogrammato (modello a memoria comune)}

\subsection{Nucleo di un sistema a processi}
In un sistema multiprogrammato, vengono offerte tante unità di elaborazione astratte quanti sono i processi. Ogni macchina possiede tutte le istruzioni elementari dell'unità centrale, più alcune irate alla gestione dei processi, alla comunicazione e sincronizzazione.

Questo modello mette in evidenza le proprietà logiche di comunicazione e sincronizzazione senza pensare agli aspetti implementativi.

\begin{mdframed}[topline=false,bottomline=false,rightline=false]
Si chiama \textbf{kernel} (nucleo) il modulo realizzato in software o firmware che supporta il concetto di processo e realizza gli strumenti necessari per la gestione di questi ultimi.
\end{mdframed}

Il nucleo è l'unico componente del sistema che è "conscio" dell'esistenza delle \textbf{interruzioni}:
ogni processo che richiede l'esecuzione di una operazione a un dispositivo effettua una chiamata primitiva al nucleo che lo \textit{sospende}, quest ultimo riceve un \textit{segnale di interruzione} dal dispositivo e risveglia il processo sospeso.

La gestione delle interruzioni è invisibile ai processi.

\subsection{Stati di un processo}

\begin{multicols}{2}
\begin{multicolfigure}
    \centering
    \includegraphics[width=\textwidth]{/home/riccardoob/appunti/sistemi_operativi/images/26.png}
\end{multicolfigure}
\columnbreak
\begin{multicolfigure}
    \centering
    \includegraphics[width=\textwidth]{/home/riccardoob/appunti/sistemi_operativi/images/27.png}
\end{multicolfigure}    
\end{multicols}

\begin{mdframed}[topline=false,bottomline=false,rightline=false]
\textbf{Contesto di un processo}\\
Insieme delle informazioni contenute nei registri del processore, quando esso opera sotto il controllo del processo.
\end{mdframed}

\begin{mdframed}[topline=false,bottomline=false,rightline=false]
\textbf{Salvataggio di contesto}\\
Quando un processo perde il controllo del processore, il contenuto dei registri del processore (contesto) viene salvato in una struttura dati associata al processo, chiamata \textbf{descrittore}.
\end{mdframed}

\begin{mdframed}[topline=false,bottomline=false,rightline=false]
\textbf{Ripristino del contesto}\\
Quando un processo viene schedulato, i valori salvati nel suo descrittore vengono caricati nei registri del processore.
\end{mdframed}

\subsection{Funzioni del nucleo}
Il compito fondamentale del nucleo è quello di \textbf{gestire le transizione di stato} dei processi, in particolare:
\begin{itemize}
    \item gestione del \textbf{salvataggio} e \textbf{ripristino} dei \textit{contesti} dei processi
    \item scelta, tra i processi pronti, del processo al quale assegnare l'unità di elaborazione (\textbf{scheduling}, secondo una particolare poitica (FIFO, SJF, Priorità etc.)
    \item gestione delle interruzioni dei dispositivi esterni
    \item realizzazione dei meccanismi di sincronizzazione dei processi
\end{itemize}

\subsection{Caratteristiche del nucleo}
\begin{mdframed}[topline=false,bottomline=false,rightline=false]
\textbf{Efficienza}\\
Dato che condiziona l'intera struttura a processi, in certi sistemi alcune funzionalità sono realizzate in hardware o microprogrammi.
\end{mdframed}

\begin{mdframed}[topline=false,bottomline=false,rightline=false]
\textbf{Dimensioni}\\
Necessità di avere un nucleo semplice e di dimensione limitata.
\end{mdframed}

\begin{mdframed}[topline=false,bottomline=false,rightline=false]
\textbf{Separazione meccanismi e politiche}\\
Il nucleo deve contenere, possibilmente, soltanto meccanismi, consentendo così a livello di processo di utilizzare tali meccanismi ad hoc.
\end{mdframed}

\section{Realizzazione del nucleo}

\subsection{Strutture dati del nucleo}

\subsubsection{Descrittore del processo}

Contiene le seguenti informazioni
\begin{itemize}
    \item \textbf{identificatore} del processo, univoco
    \item \textbf{stato} del processo: pronto, esecuzione, bloccato etc.
    \item modalità di \textbf{servizio} dei processi: parametri di scheduling, ad esempio
    \begin{itemize}
        \item FIFO
        \item Priorità
        \item Deadline (completamento richiesta processo)
        \item quanto di tempo
    \end{itemize}
    \item \textbf{contesto} del processo: \textit{contatore} di programma, \textit{registro di stato}, registri generali, indirizzo memoria privata del processo
    \item riferimenti a \textbf{code}: riferimento elemento successivo nella coda (di processi bloccati, di processi pronti...)
\end{itemize}

\begin{minted}[bgcolor=lightgray,framesep=2mm,baselinestretch=1.2,fontsize=\footnotesize]{c}
typedef struct {
    int indice_priorità;
    int delta_t;
} modalita_di_servizio;

typedef struct {
    int nome;
    ...
    modalita_di_servizio servizio;
    tipo_contesto contesto;
    tipo_stato stato;
    int successivo;
} descrittore_processo;
\end{minted}

\subsubsection{Coda dei processi pronti}
Esistono diverse code di \textit{processi pronti}, quando un processo è riattivato da un V, viene inserito in fondo alla coda corrispondente alla sua priorità.

Non sempre è presente un processo pronto, però esiste un \textbf{dummy processo} che va in esecuzione quanto tutte le altre code sono vuote, rimanendoci fino a un altro processo diventa pronto, ha la priorità più bassa ed è sempre in stato pronto.

\begin{minted}[bgcolor=lightgray,framesep=2mm,baselinestretch=1.2,fontsize=\footnotesize]{c}
typedef struct {
    int primo, ultimo;
} descrittore_coda;

typedef descrittore_coda coda_a_livelli[Npriorità];
\end{minted}

\begin{multicols}{2}
\begin{multicolfigure}
    \centering
    \includegraphics[width=\textwidth]{/home/riccardoob/appunti/sistemi_operativi/images/28.png}
\end{multicolfigure}
\columnbreak
\begin{minted}[bgcolor=lightgray,framesep=2mm,baselinestretch=1.2,fontsize=\footnotesize]{c}
void Inserimento(int P,
                 descrittore_coda C) {
    //inserimento processo di indice P
    //nella coda C
}
\end{minted}
    
\end{multicols}
\begin{multicols}{2}
\begin{multicolfigure}
    \centering
    \includegraphics[width=\textwidth]{/home/riccardoob/appunti/sistemi_operativi/images/29.png}
\end{multicolfigure}
\columnbreak
\begin{minted}[bgcolor=lightgray,framesep=2mm,baselinestretch=1.2,fontsize=\footnotesize]{c}
int Prelievo(descrittore_coda C) {
    //estrae il primo processo da coda
    //C e restituisce l'indice
}
\end{minted} 
\end{multicols}

\subsubsection{Coda dei descrittori liberi}
In questa coda sono contenuti i descrittori \textit{disponibili} per la creazione di nuovi processi, vengono reinseriti i descrittori dei processi terminati.

\begin{minted}[bgcolor=lightgray,framesep=2mm,baselinestretch=1.2,fontsize=\footnotesize]{c}
descrittore_coda descrittori_liberi;
\end{minted}

\subsubsection{Processo in esecuzione}
Il nucleo necessita di tenere traccia di quale processo è in esecuzione, informazione rappresentata dall'indice del descrittore del processo, viene salvato in una variabile del nucleo (spesso un registro del processore)
\begin{minted}[bgcolor=lightgray,framesep=2mm,baselinestretch=1.2,fontsize=\footnotesize]{c}
int processo_in_esecuzione;
\end{minted}

\subsection{Funzioni del nucleo}
Le funzioni del nucleo implementano le operazioni di \textbf{transizione di stato} per i singoli processi, ogni transizione prevede il \textit{prelievo} da una coda e l'\textit{inserimento} in un'altra.

Se la coda è vuota, essa contiene soltanto il valore -1 (NIL), valore inserito anche in code non vuote come ultimo elemento.

\subsubsection{Struttura del nucleo}
La struttura del nucleo si articola in due livelli:
\begin{itemize}
    \item \textbf{livello superiore}: contiene tutte le funzioni direttamente \textbf{utilizzate dai processi}, sia interni che esterni; in particolare le primitive per la creazione, eliminazione e sincronizzazione dei processi e le funzioni di risposta ai segnali di interruzione
    \item \textbf{livello inferiore}: realizza le funzionalità di \textbf{cambio di contesto}: salvataggio del contesto del processo deschedulato, scelta di un nuovo processo da mettere in esecuzione tra quelli pronti e ripristino del suo contesto
\end{itemize}

\subsubsection{Esecuzione del kernel}
Le funzioni del nucleo, per motivi di protezione, sono le sole che possono operare su strutture dati dello stato del sistema e utilizzare istruzioni privilegate.

Grazie al meccanismo dei \textbf{ring}, nucleo e processi eseguono in due \textbf{ambienti separati}:
\begin{itemize}
    \item \textit{nucleo}, massimo privilegio: \textbf{modo kernel} o ring 0
    \item \textit{processi}, minore privilegio: \textbf{modo user} o ring > 0
\end{itemize}

Il passaggio da un "modo" all'altro è basato sul \textit{meccanismo delle interuzioni}:
\begin{itemize}
    \item funzioni da processi \textbf{esterni}: passaggio ad ambiente nucleo tramite risposta al segnale di \textbf{interruzione}
    \item funzioni da processi \textbf{interni}: passaggio ad ambiente nucleo tramite \textbf{system calls} (SVC, interruzioni interne)
\end{itemize}

In entrambi i casi al completamento della funzione, il trasferimento avviene tramite il meccanismo di \textbf{ritorno da interruzione} (RTI).

\subsubsection{Funzioni del livello inferiore}
Le funzioni principali di \textbf{cambio di contesto} sono
\begin{itemize}
    \item \texttt{salvataggio\_stato}: salvataggio del contesto del processo in esecuzione e inserimento del descrittore nella coda dei processi bloccati o pronti
    \item \texttt{assegnazione\_cpu}: rimozione del processo a maggior priorità dalla coda e caricamento dell'identificatore nel registro del processo in esecuzione
    \item \texttt{ripristino\_stato}: caricamento del contesto del nuovo processo nei registri di macchina
\end{itemize}

\begin{minted}[bgcolor=lightgray,framesep=2mm,baselinestretch=1.2,fontsize=\footnotesize]{c}
void salvataggio_stato() {
    int j;
    j = process_in_esecuzione;
    descrittori[j].contesto = <valori dei registri CPU>;
}

void ripristino_stato() {
    int j;
    j = processo_in_esecuzione;
    <registro-temp> = descrittori[j].servizio.delta_t;
    <registri-CPU> = descrittori[j].contesto;
}

void assegnazione_CPU() { //scheduling: hp algoritmo con priorità
    int k = 0, j;
    while ((coda_processi_pronti[k].primo) == -1) {
        k++;
    }
    j = prelievo(coda_processi_pronti[k]);
    processo_in_esecuzione = j;
}
\end{minted}

\subsubsection{Gestione del temporizzatore}
Per consentire la modalità di servizio a divisione del tempo è necessario che il nucleo gestisca un \textbf{dispositivo temporizzatore} tramite una apposita procedura che ad intervalli di tempo fissati, provveda a sospendere il processo in esecuzione ed assegnare l'unità di elaborazione ad un altro processo.
















































































































